\chapter{Introduction}

Tensor Networks in the context of Quantum Error Correction have been used in two distinct places:
\begin{enumerate}
    \item Approximate decoding of QEC Codes
    \item Code construction that come with a natural tensor network decoder
\end{enumerate}



\section{Mathematical Preliminaries}
\subsection{Group Theoretic concepts}
\begin{boxed-defn}{Abelian and Non-Abelian Group}{abelian-group}
A group \(G\) is said to Abelian, iff all the elements of the group pairwise commute with each other under the group operation. It is said to be non Abelian otherwise.\\
\textbf{Examples}
\begin{enumerate}
    \item The group of integers \(\mathbb{Z}\) is Abelian under the addition operation.
    \item The set of \(2 \times 2\) matrices under matrix multiplication form a non-Abelian group.
\end{enumerate}
\end{boxed-defn}

\subsubsection{Pauli Group}
The Pauli matrices \(I, X, Y,Z\) form a group. The \(n\)-qubit Pauli group is defined as follows
\begin{boxed-defn}{n-Qubit Pauli group}{pauli-group}
\begin{equation}\label{eqn: pauli-group}
    \mathcal{G}_n = \{f: f = c\sigma_1 \otimes \sigma_2 \otimes \dots \otimes \sigma_n\} \quad \text{where} \quad \sigma_j \in \{I, X, Y, Z\} \ \text{and} \ c \in \{\pm 1, \pm i\}
\end{equation}
\end{boxed-defn}
Equivalenty single qubit Pauli operators are also defined on a index notation with the following definition
\begin{equation}\label{eqn: pauli-int-map}
    \sigma^0 = I \quad \sigma^1 = X \quad \sigma^2 = Y \quad \sigma^3 = Z
\end{equation}

\subsubsection{Stabilizer Group}

\subsubsection{Coset}

\subsubsection{Lagrange's Theorem}
This is for partitioning the stabilizer group in the disjoint subsets


\section{Stabilizer Quantum Error Correction}
\subsection{Stabilizer Codes}
We consider stabilizer quantum error correcting codes, where the codespace is determined as the subspace of the states which are simultaneously the \(+1\) eigenstates of the stabilizers. The stabilizers are a subset of the \(n\)-qubit Pauli group, \(\mathcal{G}_n\), i.e; \(\mathcal{S} \subset \mathcal{P}_n\). The codespace \(mathcal{H}_0\) can be expressed mathematically as follows
\begin{equation}\label{eqn: codespace}
    \mathcal{H}_0  = {\psi : S \ket{\psi} = \ket{\psi} \forall S \in \mathcal{S}}
\end{equation}
Consider the set of \(r\) independent generators of \(\mathcal{S}\), then \(dim(\mathcal{H}_0) = 2^(n-r)\), where \(n\) is the number of physical qubits. In other words, the number of logical qubits is \(k = n-r\).


\begin{boxed-defn}{Logical Operators of a Stabilizer Code}{stab-log-op}
The logical operators \(\mathcal{L} \subset \mathcal{G}_n\) form a non-Abelian. Given a stabilizer code \(\mathcal{S}\), it's logical operators must obey the following
\begin{enumerate}
    \item It has two sets of generators, X-type and Z-type generators, denoted as \(Z_\alpha, X_\alpha\), with \(\alpha = 1, \dots, k\)
    \item For any \(L \in \mathcal{L}\) and \(S \in \mathcal{S}\), \(LS = SL\)
    \item \(X_\alpha Z_\beta = (-1)^{\delta_{\alpha, \beta}} Z_\beta, X_\alpha\)
\end{enumerate}
\end{boxed-defn}

\begin{boxed-defn}{Destabilisers or Pure Errors}{destabs}
Another group of abelian operators \(P \in \mathcal(G)_n\) are called pure errors (or) destabilisers. They are characterized as follows
\begin{enumerate}
    \item It is a set of \(n-k\) operators \(P_i\) and \(P_i^2 = \mathbb{1}\)
    \item \(P_i S_j = (-1)^{\delta_{i,j}} S_j P_i\) for \(S_j \in \mathcal{S}\) and \(P_i \in \mathcal{P}\)
\end{enumerate}
\end{boxed-defn}

The stabilizers, destabilisers and logical operators together form the \(n\)-qubit Pauli group.

\section{Tensor Network}
\subsection{Tensors}

\subsection{MPS}

\subsection{PEPS}


\subsection{Quantum Circuits as Tensors}